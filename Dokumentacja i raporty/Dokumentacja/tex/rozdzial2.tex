\newpage
\section{Określenie wymagań szczegółowych}		%2
%Dokładne określenie wymagań aplikacji (cel, zakres, dane wejściowe) – np. opisać przyciski, czujniki, wygląd layautu, wyświetlenie okienek. Opisać zachowanie aplikacji – co po kliknięciu, zdarzenia automatyczne. Opisać możliwość dalszego rozwoju oprogramowania. Opisać zachowania aplikacji w niepożądanych sytuacjach.
\subsection{Założenia główne}
\begin{itemize}
\item \textbf{Utrzymanie modułowości projektu }
\item \textbf{Łatwość implementacji } 
\item   \textbf{Prostota w testowaniu i ewentualnym debuggingu}
\item \textbf{Podzielenie aplikacji na 2 tryby}
\item\textbf{ Użycie latarki}
\end{itemize}

\subsubsection{Utrzymanie modułowości projektu}
\hspace*{0.60cm}Pozwoli to na pracę nad wieloma "poziomami" jednocześnie co przełoży się na lepsze rozłożenie pracy pomiędzy członków grupy.

\subsubsection{Latwość implementacji}
\hspace*{0.60cm}Pozwoli to na testowanie każdego modułu osobno. Dzięki temu rozwiązaniu będziemy mogli lepiej wyeliminować błędy. A co za tym idzie lepiej dopracować nasz projekt.

\subsubsection{Prostota w testowaniu i ewentualnym debuggingu}
\hspace*{0.60cm}Chcemy dążyć do jak najłatwiejszego i jednocześnie najbardziej efektywnego sposobu testowania aplikacji. Pozwoli nam to zaoszczędzić cenny czas, który będziemy mogli poświęcić na lepsze dopracowanie szczegółów.

\subsubsection{Podzielenie aplikacji na 2 tryby}
\hspace*{0.60cm}Jednym z głównych rozwiązań w aplikacji będzie podzielenie jej na 2 zależne od siebie tryby (tryb graficzny i tekstory) zamiast tworzenia osobniej aplikacji dla każdego z trybów.
\\
\\
Jedną z wielu zalet tego rozwiązania będzie zmiejszenia nakładów pracy dzięki skupienu się tylko na jednej aplikacji. Dzięki temu aplikacja będzie bardziej dopracowana pod względem działania czy też wyglądu.
\\
\\
Drugą zaletą będzie łatwiejsze przeszukiwanie aplikacji pod względem błędów, testowanie jej czy też naprawa potencjalnych błędów, ponieważ nie trzeba będzie naprawiać tego samego błędu niekiedy w obydwu aplikacjach.
\\
\\
Trzecią zaletą będzie łatwość korzystania z aplikacji - obydwaj graczę muszą posiadać tą samą grę a nie dwie różne wersje. Dzięki temu łatwiej będzie można zamienić się trybem gry z partnerem co może zwiększyć radość z gry


\subsubsection{Użycie latarki}
Jedna z zagadek będzie oparta na wysyłaniu sygnałów w formnie kodu morsa za pomocą latarki. Latarka będzie uruchamiana na określony odstęp czasu po czym zostanie wyłączona i włączona ponownie jeżeli ostatni sygnał nie został pokazany. Kod będzie oparty na 2 sygnałach: krótkim i długim.

\subsection{Struktura aplikacji}

\hspace*{0.60cm}W aplikacji będzie dostępne:
\begin{itemize}
	\item \textbf{Menu główne}
	\item \textbf{Menu ustawień} 
	\item   \textbf{Dwa tryby gry}
\begin{itemize}
	\item \textbf{Tryb graficzny}
	\item \textbf{Tryb tekstowy} 
\end{itemize}
\end{itemize}

\subsubsection{Menu główne}
\hspace*{0.60cm}W tym panelu będziemy mieli dostęp zarówno do wyboru trybu gry jak i do ustawień
Ten panel będzie przejrzysty i łatwy w obsłudze, wszyskie opcje będą podpisane lub będą zawierały adekwatną do nazwy ikonę. Dla przykładu ustawienia zostaną oznaczone zębatką z podpisem ustawienia. Menu główne już na początku będzie zawierać jedną łatwą zagadkę

\subsubsection{Menu ustawień}
\hspace*{0.60cm}Ten panel umożliwi użytkownikom, jak sama nazwa wskazuje, ustawienia rozgrywki takie jak głośność muzyki. Do palenu ustawień będzie można wejść z każdego innego panelu. Zmiana ustawień będzie działała globalnie czyli zmiana głośności poskutuje zmianą głośności w każdym pozostałym panelu, do którego przejdziemy.

\subsubsection{Dwa tryby gry}
\hspace*{0.60cm}Użytkownicy będą mieli do wyboru tryb gry. Jeżeli użytkownik zdecyduje się na wybór trybu graficznego jedyne co będzie musiał zrobić to kliknąć w odpowieni przycisk oznaczony jako tryb graficzny. Tryby gry będą podpisane i każdy z nich będzie miał osoby panel odpowiadający za określone funkcję, które będą potrzebne do rozwiązania zagadki